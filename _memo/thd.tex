\part{Théories de la Décision}
\pagebreak

\chapter{Théorie de la décision}

La problématique est celle d'un agent qui doit prendre la meilleure décision, parmi un ensemble de choix possibles (actes), qui selon l'état du monde, mèneront à des conséquences (résultats/outcomes) différentes.\\
\\
Soient A = \{$a_1$, $...$, $a_k$ \} les actes possibles\\
Soient S = \{$s_1$, $...$, $s_m$ \} les états du mondes\\
Soient C = \{$c_1$, $...$, $c_n$ \} les conséquences\\
On n'a donc $A $ x $ S \rightarrow C$\\
Le but est de trouver le $a_i$ qui permet d'obtenir les meilleurs conséquences $c_j$.\\
\\
On distingue 3 type de théories de la décision:
\begin{description}
\item[Décisions sous certitude] il n'y a qu'une état du monde.
\item[Décision dans l'incertain] il y a plusieurs états du monde.
\item[Décision dans le risque] il y a plusieurs états du monde, dont on connait la probabilité.
\end{description}

\pagebreak

\begin{center}
$\begin{tabular}{c|c}
train & voiture\\
\hline
10 & 20 \\
\end{tabular}$
\end{center}
\begin{center}
Décision sous certitude
\end{center}
\vspace{1.5cm}
\begin{center}
$\begin{tabular}{c|cc}
\  & train & voiture\\
\hline
Normal & 10 & 20\\
Bouchon & 10 & 0 \\
\end{tabular}$
\end{center}
\begin{center}
Décision dans incertitude
\end{center}
\vspace{1.5cm}
\begin{center}
$\begin{tabular}{c|c|cc}
\  & \  & train & voiture\\
\hline
Normal & 80\% & 10 & 20\\
Bouchon & 20\% & 10 & 0 \\
\end{tabular}$
\end{center}
\begin{center}
Décision dans le risque
\end{center}

\pagebreak
\section{Décision dans l'incertain}

\subsection{Critère de Laplace}
\begin{description}
\item[] Choisir l'acte dont la conséquence moyenne est la meilleure.
\item[] $argmax_{a \in A}$ $\sum_{s \in S} \frac{1}{|A|} * u(a(s))$
\end{description}

\subsection{Critère de Wald}
\begin{description}
\item[] Choisir l'acte dont la pire conséquence est la meilleure (maximum).
\item[] $argmax_{a \in A}$ $min_{s \in S}$ $u(a(s))$
\end{description}

\subsection{Critère d'Hurwicz}
\begin{description}
\item[] Meilleur compromis entre meilleure et pire conséquences ($a \in [0,1]$)
\item[] $argmax_{a \in A}$ $( \alpha * min_{s \in S}$ $u(a(s))) + (( 1 - \alpha ) * u(a(s)))$
\end{description}

\subsection{Min Max Regret}
\begin{description}
\item[] Choisir l'acte dont on regrettera le moins les conséquences
\item[] $argmax_{a \in A}$ $max_{s \in S}$ $R(a,s)$ avec $R(a,s) = max_{b \in A} u(b(s)) - u(a(s))$
\end{description}

\subsection{Example}

\begin{tabular}{c|ccc|cccc}
\hline
Actes & Etats & du & monde & $ $ & $ $ & $ $ & $ $\\
\hline
$ $  & $s_1$ & $s_2$ & $s_3$ & $Laplace$ & $Wald$ & $Hurwicz_{.5}$ & $MinMax Regret$ \\
\hline
$a_1$ & $55_{21}$ & $10_{12}$ & $13_{13}$ & $26$ & $10$ & $34$ & $\crouge{21}$\\
$a_2$ & $40_{36}$ & $19_3$ & $22_4$ & $\crouge{27}$ & $19$ & $31$ & $36$\\
$a_3$ & $30_{48}$ & $20_0$ & $26_0$ & $26$ & $\crouge{22}$ & $28$ & $46$\\
$a_4$ & $76_0$ & $2_{20}$ & $0_{26}$ & $26$ & $0$ & $\crouge{38}$ & $26$\\
\hline
\end{tabular}

\pagebreak

\subsection{Différents cadres d'incertitude}
\begin{description}
\item[Décision dans le risque (incertitude probabiliste)]: MinMax Regret
\item[Décision dans l'incertain (incertitude qualitative)]: Prade
\item[Décision sous incertitude stricte]: Wald, Hurwicz
\item[Décision sous ignorance total]: Konieczny, Marquis
\end{description}

\pagebreak
\chapter{Théorie des jeux}
\pagebreak

\section{Jeux sous forme stratégique}

Un jeu sous forme stratégique est défini par:
\begin{description}
\item[] un ensemble $N$ = $\{1,....,n\}$ de joueurs
\item[] pour chaque joueurs $i$ un ensemble de stratégies $S_i$ = $\{s_1,.....,S_{n_i}\}$
\item[] pour chaque joueurs $i$ une fonction de valuation $u_i : S_1$ x ... x $S_n \rightarrow R_i$ qui pour un ensemble de stratégies associe les gains du joueur $i$
\end{description}

On notera:\\
\begin{description}
\item[$s$] un profil de stratégies $\{s_1,...,s_n\}$ où $\forall i s_i \in S_i$
\item[$s_{-i}$] le profil $s$ des stratégies autre que celle du joueurs $i$
\item[$S$] l'espace des stratégies 
\end{description}

\subsection{utilité}
On appelle utilité la mesure de chaque situation aux yeux de l'agent, celle ci n'est si une mesure du gain matériel, monétaire, etc, mais une mesure subjective du contentement de l'agent.
\pagebreak
\subsection{jeux sous forme extensive et stratégique}

\begin{center}
\begin{tabular}{c|cc}
$ $ & u & v\\
\hline
x & 4,2 & 3,1\\
y & 2,5 & 9,0\\
\end{tabular}
\end{center}
\begin{center}
Forme stratégique
\end{center}
\ \\
x et y étant les choix représenté par le joueur 1.\\
u et v étant les choix représenté par le joueur 2.\\
Si le joueur 1 choisis x et le joueur 2 v alors le joueur 1 gagnera 3 et le joueur 2 gagnera 1.\\

\begin{center}
\begin{tikzpicture}[->,>=stealth',shorten >=1pt,auto,node distance=3cm,
                    semithick]
  \tikzstyle{every state}=[fill=white,draw=none,text=black]

  \node[state]         (A)                    {$A$};
  \node[state]         (B) [below left of=A]  {$B$};
  \node[state]         (C) [below right of=A] {$C$};
  \node[state]         (XU)[below left of=B]  {$(4,2)$};
  \node[state]         (XV) [below of=B]      {$(3,1)$};
  \node[state]		   (YU) [below of=C]      {$(2,5)$};
  \node[state]    	   (YV)[below right of=C] {$(9,0)$};

  \path (A) edge 			  node {x} (B)
  			edge			  node {y} (C)
  		(B) edge 			  node {u} (XU)
  		    edge 			  node {v} (XV)
  		(C) edge 			  node {u} (YU)
  		    edge			  node {v} (YV);
\end{tikzpicture}
\end{center}
\begin{center}
Forme Extensive
\end{center}

\pagebreak
\subsection{Élimination de stratégies dominées}

Une stratégie $s_i$ est (strictement) dominé pour le joueur $i$ si il existe une stratégie $s_i'$ telle que pour tout profil $s_{-i}$ 
\begin{description}
\item[] $u_i(s_i', s_{-i}) > u_i ( s_i, s_{-i})$
\end{description}

Une stratégie faiblement dominé est sous la forme:
\begin{description}
\item[] $u_i(s_i', s_{-i}) \geq u_i ( s_i, s_{-i})$
\end{description}

\begin{multicols}{3}
[]
\begin{tabular}{c|cc}
$ $ & u & v\\
\hline
x & 4,2 & 3,1\\
y & 2,5 & 9,0\\
\end{tabular}

\begin{tabular}{c|cc}
$ $ & u & $\crouge{v}$\\
\hline
x & 4,2 & $\crouge{3,1}$\\
y & 2,5 & $\crouge{9,0}$\\
\end{tabular}

\begin{tabular}{c|cc}
$ $ & u & $\crouge{v}$\\
\hline
x & 4,2 & $\crouge{3,1}$\\
$\crouge{y}$ & $\crouge{2,5}$ & $\crouge{9,0}$\\
\end{tabular}
\end{multicols}

Le profil (4,2) est sélectionné donc Joueur 1 gagnera 4 et Joueur 2 gagnera 2.\\

\subsection{Équilibre de Nash}

Un jeu peut avoir plusieurs ou aucun équilibre de Nash.\\
\begin{tabular}{c|ccc}
$ $ & u & v & w\\
\hline
x & 3,0 & 0,2 & 0,3\\
y & 2,0 & $\crouge{1,1}$ & 2,0\\
z & 0,3 & 0,2 & 3,0\\
\hline
\end{tabular}
\ \\\\
Deux équilibre de Nash sont interchangeable si la permutation des termes gauche garde l'équilibre de Nash actif.\\

Voici un cas particulier:
\begin{tabular}{c|cc}
$ $ & u & v\\
\hline
x & 2,1 & 0,0\\
y & 0,0 & 1,2\\
\end{tabular}

Deux équilibre de Nash sont présent $\crouge{(2,1)}$ et $\crouge{(1,2)}$.
Comme il y a une hésitation entre les deux cas, alors l'utilisation du flip coin est envisageable:
\begin{description}
\item[] $u_1(<(x, 1/2), (y, 1/2)>) = 1/2 * 2 + 1/2 * 0 = 1$
\item[] $u_1(<(u, 1/2), (v, 1/2)>) = 1/2 * 0 + 1/2 * 1 = 1/2$
\end{description}

\subsection{Critère de Pareto}

\begin{multicols}{2}
[Soit la table:]

\begin{tabular}{c|cc}
$ $ & u & v\\
\hline
x & 4,4 & 3,1\\
y & 2,3 & 7,5\\
\end{tabular}
\vspace{1.5cm}

\begin{description}
\item[] Pour u, x est meilleur que y
\item[] Pour v, y est meilleur que x
\item[] Pour x, u est meilleur que v
\item[] Pour y, v est meilleur que u
\end{description}

\end{multicols}

\begin{multicols}{2}
[Un profil s domine un profil s' dans le sens de Pareto si pour tout les joueurs s est au moins meilleur que s' et que pour un joueur s est meilleur strictement que s'.\\
Un profil s domine strictement un profil s' dans le sens de Pareto si pour tout les joueurs s est meilleur que s'.\\
]

\begin{tabular}{c|cc}
$ $ & u & v\\
\hline
x & 4,4 & $\crouge{3,1}$\\
y & $\crouge{2,3}$ & 7,5\\
\end{tabular}

\begin{description}
\item[] (x,u) 4,4
\item[] (y,v) 7,5 est meilleur
\end{description}

\end{multicols}

\subsection{Niveau de sécurité}
Pour un tableau:

\begin{tabular}{c|cc}
$ $ & u & v\\
\hline
x & 9,9 & 0,8\\
y & 8,0 & 7,7\\
\end{tabular}
\ \\\\
Dans le cas d'un jeu avec des joueurs non rationnel, l'un des deux joueur peut duper l'autre et ainsi gagner 8 et faire gagner 0 à l'autre joueur. \\\\
On défini le niveau de sécurité d'une stratégie $s_i$ pour le joueur i comme le gain minimum que peut apporter cette stratégie quel que soit le choix des autres joueurs.\\
On défini le niveau de sécurité d'un joueur i comme le niveau de sécurité maximal des stratégies de i.\\

Le meilleur choix serait de prendre (y,v) pour assurer un minimum de gain pour chaque personnes.\\

\subsection{autres Stratégies}
Jusque la nous avons utilisé que les stratégies pures, c'est à dire les option qui se présente au joueurs.\\
Une stratégies mixte est une distribution de probabilité sur l'ensemble des stratégies pures.\\
%% comportemental / local

\subsection{Équilibre de Nash en stratèges mixtes}
Soit le problème:
\begin{tabular}{cc|cc}
$ $ &$ $ & y & 1-y\\
\hline
$ $& $ $ & f & c\\
\hline
x & f & 2,1 & 0,0\\
1-x & c & 0,0 & 1,2\\
\end{tabular}

Soit y, la probabilité avec laquelle le joueur 2 jour f, quelle est la meilleure réponse du joueur 1?:
\begin{description}
\item[] $u_1(<(f, y), (c, 1-y)>) = y*2 + (1-y)*0 = 2y$
\item[] $u_1(<(f, y), (c, 1-y)>) = y*0 + (1-y)*1 = 1-y$
\end{description}

Donc:
\begin{description}
\item[] Si $2y > 1 - y$ avec ($y > 1/3$), la meilleur réponse du joueur 1 est de jouer $f$
\item[] Si $2y < 1 - y$ avec ($y < 1/3$), la meilleur réponse du joueur 1 est de jouer $c$
\item[] Si $2y = 1 - y$ avec ($y = 1/3$), le joueur 1 peut jouer l'un ou l'autre.
\end{description}

Soit x, la probabilité avec laquelle le joueur 1 jour f, quelle est la meilleure réponse du joueur 2?:
\begin{description}
\item[] $u_1(<(f, x), (c, 1-x)>) = x*1+(1-x)*0 = x$
\item[] $u_1(<(f, x), (c, 1-x)>) = x*1+(1-x)*0 = 2(1-x)$
\end{description}

Donc:
\begin{description}
\item[] Si $x > 2(1-x)$ avec ($x > 2/3$), la meilleur réponse du joueur 2 est de jouer $f$
\item[] Si $x < 2(1-x)$ avec ($x < 2/3$), la meilleur réponse du joueur 2 est de jouer $c$
\item[] Si $x = 2(1-x)$ avec ($x = 2/3$), le joueur 2 peut jouer l'un ou l'autre.
\end{description}

%%fill

\subsection{Représentation graphique du jeu}
%% fill

\subsection{Coopération}

\begin{tabular}{c|cc}
$ $ & f & c\\
\hline
f & 2,1 & 0,0\\
& c & 0,0 & 1,2\\
\end{tabular}

Que se passe t'il si les 2 joueurs peuvent communiquer avant de jouer?:
\formula{$u_1 = u_2 = \frac{1}{2} * 2 + \frac{1}{2} = \frac{3}{2}$}

Lorsque tout les joueurs peuvent observer un même événement aléatoire, ils peuvent alors s'accorder sur des équilibres corrélés.\\
Selon un accord prit via un flip coin, ou via une parution d'un évènement, si les deux joueurs se mette d'accord sur le fait de tirer $f$ ou $c$, mais le joueur désavantagé peut ne pas jouer le choix prit, mais il s'expose à ne rien gagner.\\

\subsection{Itération}
\begin{tabular}{c|cc}
$ $ & c & d\\
\hline
c & 3,3 & 0,5\\
d c & 5,0 & 1,1\\
\end{tabular}

%% fill [DIP]	

\pagebreak